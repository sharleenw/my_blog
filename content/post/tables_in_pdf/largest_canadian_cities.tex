\documentclass[]{article}
\usepackage{lmodern}
\usepackage{amssymb,amsmath}
\usepackage{ifxetex,ifluatex}
\usepackage{fixltx2e} % provides \textsubscript
\ifnum 0\ifxetex 1\fi\ifluatex 1\fi=0 % if pdftex
  \usepackage[T1]{fontenc}
  \usepackage[utf8]{inputenc}
\else % if luatex or xelatex
  \ifxetex
    \usepackage{mathspec}
  \else
    \usepackage{fontspec}
  \fi
  \defaultfontfeatures{Ligatures=TeX,Scale=MatchLowercase}
\fi
% use upquote if available, for straight quotes in verbatim environments
\IfFileExists{upquote.sty}{\usepackage{upquote}}{}
% use microtype if available
\IfFileExists{microtype.sty}{%
\usepackage{microtype}
\UseMicrotypeSet[protrusion]{basicmath} % disable protrusion for tt fonts
}{}
\usepackage[left=1.5cm,right=7cm,top=2cm,bottom=2cm]{geometry}
\usepackage{hyperref}
\hypersetup{unicode=true,
            pdfborder={0 0 0},
            breaklinks=true}
\urlstyle{same}  % don't use monospace font for urls
\usepackage{longtable,booktabs}
\usepackage{graphicx,grffile}
\makeatletter
\def\maxwidth{\ifdim\Gin@nat@width>\linewidth\linewidth\else\Gin@nat@width\fi}
\def\maxheight{\ifdim\Gin@nat@height>\textheight\textheight\else\Gin@nat@height\fi}
\makeatother
% Scale images if necessary, so that they will not overflow the page
% margins by default, and it is still possible to overwrite the defaults
% using explicit options in \includegraphics[width, height, ...]{}
\setkeys{Gin}{width=\maxwidth,height=\maxheight,keepaspectratio}
\IfFileExists{parskip.sty}{%
\usepackage{parskip}
}{% else
\setlength{\parindent}{0pt}
\setlength{\parskip}{6pt plus 2pt minus 1pt}
}
\setlength{\emergencystretch}{3em}  % prevent overfull lines
\providecommand{\tightlist}{%
  \setlength{\itemsep}{0pt}\setlength{\parskip}{0pt}}
\setcounter{secnumdepth}{0}
% Redefines (sub)paragraphs to behave more like sections
\ifx\paragraph\undefined\else
\let\oldparagraph\paragraph
\renewcommand{\paragraph}[1]{\oldparagraph{#1}\mbox{}}
\fi
\ifx\subparagraph\undefined\else
\let\oldsubparagraph\subparagraph
\renewcommand{\subparagraph}[1]{\oldsubparagraph{#1}\mbox{}}
\fi

%%% Use protect on footnotes to avoid problems with footnotes in titles
\let\rmarkdownfootnote\footnote%
\def\footnote{\protect\rmarkdownfootnote}

%%% Change title format to be more compact
\usepackage{titling}

% Create subtitle command for use in maketitle
\providecommand{\subtitle}[1]{
  \posttitle{
    \begin{center}\large#1\end{center}
    }
}

\setlength{\droptitle}{-2em}

  \title{}
    \pretitle{\vspace{\droptitle}}
  \posttitle{}
    \author{}
    \preauthor{}\postauthor{}
    \date{}
    \predate{}\postdate{}
  
\usepackage{booktabs}
\usepackage{longtable}
\usepackage{array}
\usepackage{multirow}
\usepackage{wrapfig}
\usepackage{float}
\usepackage{colortbl}
\usepackage{pdflscape}
\usepackage{tabu}
\usepackage{threeparttable}
\usepackage{threeparttablex}
\usepackage[normalem]{ulem}
\usepackage{makecell}
\usepackage{xcolor}

\begin{document}

\section{The top 100 cities in
Canada}\label{the-top-100-cities-in-canada}

Table \ref{tab:table-population-by-province} shows the populations of
each province and of Canada in total for the years 2011 and 2016 (this
data includes only the top 100 largest cities in Canada).

\begin{table}[H]

\caption{\label{tab:table-population-by-province}Number of cities and population of each province (including only the largest 100 cities in Canada)}
\centering
\begin{threeparttable}
\begin{tabular}{>{\raggedright\arraybackslash}p{6cm}>{\raggedleft\arraybackslash}p{3cm}>{\raggedleft\arraybackslash}p{3cm}>{\raggedleft\arraybackslash}p{3cm}}
\toprule
\multicolumn{1}{c}{\cellcolor[HTML]{9BD4F5}{\textbf{ }}} & \multicolumn{1}{c}{\cellcolor[HTML]{9BD4F5}{\textbf{ }}} & \multicolumn{2}{c}{\cellcolor[HTML]{9BD4F5}{\textbf{Population}}} \\
\rowcolor[HTML]{9BD4F5}  \multicolumn{1}{>{\centering\arraybackslash}p{6cm}}{\textbf{Province}} & \multicolumn{1}{>{\centering\arraybackslash}p{3cm}}{\textbf{Number of largest 100 cities in this province}} & \multicolumn{1}{>{\centering\arraybackslash}p{3cm}}{\textbf{2011}} & \multicolumn{1}{>{\centering\arraybackslash}p{3cm}}{\textbf{2016}}\\
\midrule
Alberta & 13 & 2,538,547 & 2,891,712\\
\rowcolor[HTML]{DDDDDD}  British Columbia & 14 & 3,254,203 & 3,471,292\\
Manitoba & 2 & 715,649 & 760,249\\
\rowcolor[HTML]{DDDDDD}  New Brunswick & 4 & 244,910 & 250,811\\
Newfoundland and Labrador & 1 & 172,312 & 178,427\\
\rowcolor[HTML]{DDDDDD}  Nova Scotia & 2 & 335,154 & 346,605\\
Ontario & 39 & 10,140,286 & 10,659,522\\
\rowcolor[HTML]{DDDDDD}  Prince Edward Island & 1 & 41,613 & 44,739\\
Quebec & 20 & 5,140,554 & 5,337,846\\
\rowcolor[HTML]{DDDDDD}  Saskatchewan & 4 & 479,228 & 527,638\\
\rowcolor[HTML]{FF9000}  \textbf{Canada} & \textbf{100} & \textbf{23,062,456} & \textbf{24,468,841}\\
\bottomrule
\end{tabular}
\begin{tablenotes}
\item \textit{Footnote:} 
\item There are other cities in Canada not included in this table; the excluded cities are those smaller than the 100 largest cities in Canada.
\end{tablenotes}
\end{threeparttable}
\end{table}


\end{document}
